% Created 2017-03-17 Fri 18:45
\documentclass[bigger]{beamer}
\usepackage[utf8]{inputenc}
\usepackage[T1]{fontenc}
\usepackage{fixltx2e}
\usepackage{graphicx}
\usepackage{longtable}
\usepackage{float}
\usepackage{wrapfig}
\usepackage{rotating}
\usepackage[normalem]{ulem}
\usepackage{amsmath}
\usepackage{textcomp}
\usepackage{marvosym}
\usepackage{wasysym}
\usepackage{amssymb}
\usepackage{hyperref}
\tolerance=1000
\usepackage[german, germanb]{babel}
\uselanguage{german}
\usepackage{url}
\usepackage{pgfplots}
\usetikzlibrary{pgfplots.groupplots}
\pdfmapfile{+sansmathaccent.map}
\mode<beamer>{\usetheme{Berkeley}}
\usetheme{default}
\author{Stephan Messlinger \\ Valentin Ochs}
\date{2017-03-20}
\title{Blinkenlights Workshop}
\hypersetup{
  pdfkeywords={},
  pdfsubject={},
  pdfcreator={Emacs 24.5.1 (Org mode 8.2.10)}}
\begin{document}

\maketitle

\section{Digital Out}
\label{sec-1}
\begin{frame}[fragile,label=sec-1-1]{Startpunkt digitaler Output}
 Blink Beispiel: File $\rightarrow$ Examples $\rightarrow$ Basics $\rightarrow$ Blink

\begin{verbatim}
void setup() {
  pinMode(13, OUTPUT);
}

void loop() {
  digitalWrite(13, HIGH);
  delay(1000);
  digitalWrite(13, LOW);
  delay(1000);
}
\end{verbatim}
\end{frame}

\begin{frame}[fragile,label=sec-1-2]{Setup}
 \verb~pinMode(pin, modus)~ wählt für den Pin mit Nummer \verb~pin~ eine von drei
Betriebsarten:

\begin{itemize}
\item \verb~OUTPUT~: wird für Ausgabe verwendet, z.B. um LEDs zu schalten oder
mit anderen Microcontrollern zu sprechen.
\item \verb~INPUT~: Die Spannung am Pin kann gelesen werden.
\item \verb~INPUT_PULLUP~: Wie \verb~INPUT~, aber der Pin wird intern auf die
Versorgunsspannung gezogen.
\end{itemize}

TODO: Bild zu Pullups?
\end{frame}

\begin{frame}[fragile,label=sec-1-3]{digitalWrite und delay}
 \verb~digitalWrite(pin, zustand)~ setzt bei einem auf Output gestellten Pin
die Ausgangsspannung:

\begin{itemize}
\item 0 Volt für \verb~LOW~
\item 5 Volt für \verb~HIGH~ (oder was auch immer die aktuelle
Versorgungsspannung ist)
\end{itemize}

\verb~delay(ms)~ tut \verb~ms~ Millisekunden lang nichts.
\end{frame}

\begin{frame}[fragile,label=sec-1-4]{Andere Blink Muster}
 Zwei Sekunden lang an, eine halbe aus.

\pause

\begin{verbatim}
digitalWrite(13, HIGH);
delay(2000);
digitalWrite(13, LOW);
delay(500);
\end{verbatim}
\end{frame}

\begin{frame}[label=sec-1-5]{Mehrere LEDs}
hier einfuegen?

Timer + if
\end{frame}

\begin{frame}[label=sec-1-6]{Schnelleres Blinken}
auf 1 ms oder so setzen, Verhaeltnisse aendern
\end{frame}

\section{PWM}
\label{sec-2}
\begin{frame}[fragile,label=sec-2-1]{analogWrite}
 \verb~analogWrite(pin, wert)~ schaltet den Pin automatisch an und aus, mit
variablen An-/Aus-Zeiten $\rightarrow$ Pulsweitenmodulation

Frequenz: Etwa 490 Hz

Wertebereich: 0 bis 255

Nur auf Pins 3, 5, 6, 9, 10, und 11. Die PWM Pins sind auf dem Arduino
mit \textasciitilde{} markiert.
\end{frame}

\begin{frame}[label=sec-2-2]{PWM Funktionsweise: Zähler + Vergleich}
\begin{tikzpicture}
\begin{axis}[xlabel=Zeit / s, ylabel=Zähler, ymin=-0.02*256, ymax=1.02*256]
\addplot[blue, domain=0:0.001, samples=512] { floor(mod(x*490*2*pi*256, 256)) };
\addplot[red, domain=0:0.001, samples=2] { 128 };
\end{axis}
\end{tikzpicture}
\end{frame}

\begin{frame}[label=sec-2-3]{PWM, Schwellwert 128}
\begin{tikzpicture}
\begin{axis}[xlabel=Zeit / s, ylabel=Spannung / V, ymin=-0.1, ymax=5.1]
\addplot[blue, domain=0:0.001, samples=500] { 5*ceil(0.5-mod(x*490*2*pi, 1)) };
\addplot[red, domain=0:0.001, samples=2] { 2.5 };
\end{axis}
\end{tikzpicture}
\end{frame}

\begin{frame}[label=sec-2-4]{PWM, Schwellwert 16}
\begin{tikzpicture}
\begin{axis}[xlabel=Zeit / s, ylabel=Spannung / V, ymin=-0.1, ymax=5.1]
\addplot[blue, domain=0:0.001, samples=500] { 5*ceil(0.0625-mod(x*490*2*pi, 1)) };
\addplot[red, domain=0:0.001, samples=2] { 16./256 };
\end{axis}
\end{tikzpicture}
\end{frame}

\begin{frame}[fragile,label=sec-2-5]{Einfacher PWM Code}
 \begin{verbatim}
void setup() {
  pinMode(11, OUTPUT);
}
void loop() {
  // Zeit seit Beginn des Programms
  unsigned long time = millis();
  // Berechne einen Sinus mit 0.1 Hz
  int value = 127.5 *
   (1+sin(millis * 0.001 * 0.1 * 2 * 3.1416));
  // Verwende den Wert als Schwellwert
  analogWrite(11, value);
}
\end{verbatim}
\end{frame}

\begin{frame}[label=sec-2-6]{PWM Frequenz}
490 Hz sind bei schnellen Bewegungen sichtbar.

Bestimmung der Frequenz: Taktfrequenz / Vorteiler / Maximalwert

Taktfrequenz: 16 MHz
Maximalwert: 256 für Pins 5 und 6, 510 für 3, 9, 10, 11
\end{frame}

\begin{frame}[fragile,label=sec-2-7]{PWM Vorteiler: Timer 0, Pins 5 und 6}
 \begin{center}
\begin{tabular}{rrr}
Einstellung & Teiler & Frequenz\\
\hline
0x01 & 1 & 62500\\
0x02 & 8 & 7813\\
0x03 & 64 & 977\\
0x04 & 256 & 244\\
0x05 & 1024 & 61\\
\end{tabular}
\end{center}

Einstellen durch
\begin{verbatim}
TCCR0B = (TCCR0B & 0b11111000) | Einstellung
\end{verbatim}
\end{frame}

\begin{frame}[fragile,label=sec-2-8]{PWM Vorteiler: Timer 1, Pins 9 und 10}
 \begin{center}
\begin{tabular}{rrr}
Einstellung & Teiler & Frequenz\\
\hline
0x01 & 1 & 31373\\
0x02 & 8 & 3921\\
0x03 & 64 & 490\\
0x04 & 256 & 123\\
0x05 & 1024 & 31\\
\end{tabular}
\end{center}

Einstellen durch
\begin{verbatim}
TCCR1B = (TCCR0B & 0b11111000) | Einstellung
\end{verbatim}
\end{frame}

\begin{frame}[fragile,label=sec-2-9]{PWM Vorteiler: Timer 2, Pins 11 und 3}
 \begin{center}
\begin{tabular}{rrr}
Einstellung & Teiler & Frequenz\\
\hline
0x01 & 1 & 31373\\
0x02 & 8 & 3921\\
0x03 & 32 & 980\\
0x04 & 64 & 490\\
0x05 & 128 & 245\\
0x06 & 256 & 123\\
0x07 & 1024 & 31\\
\end{tabular}
\end{center}

Einstellen durch
\begin{verbatim}
TCCR2B = (TCCR2B & 0b11111000) | Einstellung
\end{verbatim}
\end{frame}

\begin{frame}[fragile,label=sec-2-10]{Vorsicht}
 Frequenzänderung beeinflusst nicht nur LEDs, sondern alles, was an dem
Timer hängt! Servos, Tonerzeugung, etc.

Besonders wichtig: Timer 0 für \verb~millis()~ und
\verb~delay()~. Standardvorteiler: 64. Bei Änderungen Zeiten entsprechend
anpassen (Vervierfachen bei 256\ldots{})
\end{frame}

\section{Digital In}
\label{sec-3}
\begin{frame}[label=sec-3-1]{Startpunkt digitaler Input}
Button Beispiel: File $\rightarrow$ Examples $\rightarrow$ Digital $\rightarrow$ Blink
\end{frame}

\begin{frame}[label=sec-3-2]{Schaltplan}
\begin{itemize}
\item pulldown, extern
\item pullup, extern
\item pullup, intern
\end{itemize}
\end{frame}

\begin{frame}[label=sec-3-3]{Unterbrechbare Abläufe starten}
millis() und so?
\end{frame}
% Emacs 24.5.1 (Org mode 8.2.10)
\end{document}
